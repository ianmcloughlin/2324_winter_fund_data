\documentclass{iansnotes}

\title{Tasks: Fundamentals of Data Analysis}
\author{ian.mcloughlin@atu.ie}
\date{Last updated: \today}

\begin{document}
 
\maketitle
 
\begin{enumerate}
  
%  \item The Fibonacci numbers are defined as follows. The first number, $f_0$ is 0 and the second, $f_1$ is $1$. The other numbers are then defined as
%  $$f_i = f_{i-1} + f_{i-2}$$
%  for $i$ from $2$ to infinity.
  
%  In this task, write a function \mintinline{python}{fib(n)} to calculate the $n^{\textrm{th}}$ Fibonacci number.

  \item The Collatz conjecture~\autocite{quantacollatz} is a famous unsolved problem in mathematics. The problem is to prove that if you start with any positive integer $x$ and repeatedly apply the function $f(x)$ below, you always get stuck in the repeating sequence $1,2,4,1,2,4,\ldots$ 
  
  $$ f(x) = \begin{cases}
    x \div 2 & \text{if } x \text{ is even} \\
    3x + 1              & \text{otherwise} 
  \end{cases}$$

  For example, starting with the value 10, which is an even number, we divide it by 2 to get 5.
  Then 5 is an odd number so, we multiply by 3 and add 1 to get 16.
  Then we repeatedly divide by 2 to get 8, 4, 2, 1.
  Once we are at 1, we go back to 4 and get stuck in the repeating sequence $4, 2, 1$ as we suspected.

  Your task is to verify, using Python, that the conjecture is true for the first 10,000 positive integers.
  
\end{enumerate}

\end{document} 