\documentclass{iansnotes}

\title{Assessment}
\author{Fundamentals of Data Analysis, Winter 2023/24\\ian.mcloughlin@atu.ie}
\date{Last updated: \today}

\begin{document}
 
\maketitle

These are the instructions for the assessment of Fundamentals of Data Analysis in Winter 2023/24.
They cover the full 100\% of the module marks.
The deadline for submission of all components within the repository is \textbf{Sunday 7 January 2024}.

\section{Purpose}
The main purpose of this assessment is to ensure students can demonstrate they can do the following.
\begin{enumerate}
  \item Source and investigate sets of data.
  \item Programmatically explore and visualize data.
  \item Apply basic mathematical data analysis techniques to data sets.
  \item Write programs to automate basic data analysis techniques.
\end{enumerate}
The assessment is then used to determine a mark out of 100\% based on the details in this document.


\section{Instructions}

\begin{itemize}
  \item There are three components to this assessment.
  \begin{itemize}
    \item Tasks throughout the semester (40\%).
    \item A project (40\%).
    \item A repository containing all above work (20\%).
  \end{itemize}
  \item All work must be in the \texttt{main} branch of one GitHub repository\footnote{If you have any issues with GitHub branches please ask for help long before the deadline. Note that GitHub renamed their default \texttt{master} branch to \texttt{main} some time ago. This should not cause issues if you create your repository as described in lectures.}.
  \item Use the form on the module page to submit your repository\footnote{You should set up the repository and submit the URL immediately.}.
  \item Commits in GitHub on or before the deadline will be considered\footnote{Once you have submitted your URL, you do not need to do anything other than commit to your repository and push the changes to GitHub.}.
  \item You start immediately, spreading the work throughout the semester. This will ensure you make an regular commits to your repository.
  \item Your repository should contain the following.
  \begin{itemize}
    \item A \texttt{README.md} giving an executive summary of the purpose and contents of the repository, and instructions for a competent person to clone and run any code or notebooks in it.
    \item A \texttt{.gitignore} file to ignore any temporary files and folders that should not normally be committed to a repository.
    \item A notebook\footnote{All uses of the term notebook in this document refer to Jupyter notebooks, as used in lectures.} called \texttt{tasks.ipynb} containing work as detailed in the Tasks section below.
    \item A notebook called \texttt{project.ipynb} containing work as detailed in the Project section below.
    \item Any supporting data, image, or other files that form part of your submission. These should be neatly arranged, using subfolders as appropriate.
  \end{itemize}
  \item Once completed, your repository should be readily presentable in job interviews. A technically competent person should be able to understand what is in your repository and how to interact with it without any extra information. This criterion will be decisive in determining your repository mark (20\%).
\end{itemize}


\section{Tasks}

\begin{itemize}
  \item Each topic covered in lectures will have tasks associated with it.
  \item The tasks will be listed in an accompanying document.
  \item All tasks for all topics should be completed in a single notebook called \texttt{tasks.ipynb}.
  \item There will be five topics, about one every two weeks throughout the semester.
  \item The first topic will cover getting set up with the technology used in the module.
  \item The tasks should be largely completed and committed to your repository as they are covered or soon after.
  \item Any changes to the tasks during the semester will be flagged by the lecturer. Please double-check the tasks list near the deadline to ensure you have completed all of them.
\end{itemize} 


\section{Project}

\begin{itemize}
  \item The project is to create a notebook investigating the variables and data points within the well-known iris flower data set associated with Ronald A Fisher~\autocite{irisdataset}. 
  \item In the notebook, you should discuss the classification of each variable within the data set according to common variable types and scales of measurement in mathematics, statistics, and Python.
  \item Select, demonstrate, and explain the most appropriate summary statistics to describe each variable.
  \item Select, demonstrate, and explain the most appropriate plot(s) for each variable.
  \item The notebook should follow a cohesive narrative about the data set.
\end{itemize} 


\section{Marking Scheme}
Each of the three components of your submission will be marked using the four categories below.
To receive a good mark in a category, your submission needs to provide evidence of meeting each of the criteria listed under it\footnote{In line with ATU policy, the examiners' overall impression of the submission may affect marks in each category.}.

\begin{description}
  \item[Research $(25\%)$:] evidence of research on topics; appropriate referencing; building on work of others; comparison to similar work.
  \item[Development $(25\%)$:] clear, concise, and correct code; appropriate tests; demonstrable knowledge of different approaches and algorithms; clean architecture.
  \item[Documentation $(25\%)$:] clear explanations of concepts in notebooks; concise comments in code and elsewhere; appropriate, standard README for a GitHub repository.
  \item[Consistency $(25\%)$:] tens of commits, each representing a reasonable amount of work; literature, documentation, and code evidencing work on the assessment; evidence of reviewing and refactoring.
\end{description}

\section{Policies}

\begin{itemize}
  \item Please remember that you are bound by ATU policies and regulations~\autocite{atupolicies}. You should familiarize yourself with these on the Student Hub~\autocite{atustudenthub}.
  \item Pay particular attention to the Policy on Plagiarism and the Student Code of Conduct.
  \item If you have any doubts about what is permissible, email me to ask\footnote{\url{ian.mcloughlin@atu.ie}}.
\end{itemize}

\section{Advice}

\begin{itemize}
  \item Students sometimes struggle with the freedom given in an open-style assessment.
  \item You must decide where and how to start, what is relevant content for your submission, how much is enough, and how to make the submission your own.
  \item This is by design --- we assume you have a reasonable knowledge of programming and an ability to source your own information.
  \item Companies tell us they want graduates who can (within reason) take initiative, work independently, source information, and make design decisions without needing to ask for help.`'
  \item You need a plan, you cannot just start coding straight away.
\end{itemize}

\end{document}